\documentclass[a4paper,12pt]{article}

% Packages
\usepackage[utf8]{inputenc}
\usepackage{amsmath, amssymb}
\usepackage{graphicx}
\usepackage{float}
\usepackage{hyperref}
\usepackage{geometry}
\geometry{margin=1in}

% Title and Author
\title{Modern Physics Experiment Report: Zeeman Effect}
\author{Henry Su, Wen Hong Lou\\ NTU Physics Department}
\date{\today}

\begin{document}

\maketitle

\begin{abstract}
This report investigates the Zeeman Effect, a phenomenon where spectral lines are split into multiple components in the presence of a magnetic field. The experiment aims to measure the splitting and verify the theoretical predictions. 
\end{abstract}

\tableofcontents
\newpage

\section{Introduction}
The Zeeman Effect, discovered by Pieter Zeeman, is a crucial phenomenon in modern physics that demonstrates the interaction between magnetic fields and atomic energy levels. This section introduces the theoretical background and significance of the Zeeman Effect.

\section{Theory}
\subsection{Zeeman effect}
The Zeeman effect is the spliting of hyper energy state with the presence of external magnetic field. We can denote the magnetic displacement as 
$H = \mu B$

\section{Experimental Setup}
THIs is me.
Describe the f used, including the spectrometer, light source, and magnetic field generator. Include a diagram if possible:

\section{Procedure}
Outline the steps taken to perform the experiment, including calibration, data collection, and analysis.

\section{Results}
Present the observed spectral line splitting and compare it with theoretical predictions. Include tables and graphs where necessary.

\section{Discussion}
Analyze the results, discuss sources of error, and evaluate the agreement between experimental and theoretical values.

\section{Conclusion}
Summarize the findings and their implications for understanding the Zeeman Effect.

\section*{References}
List all references used in the report, formatted appropriately.

\end{document}